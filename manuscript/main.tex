\documentclass[fleqn,10pt]{wlscirep}
\usepackage[utf8]{inputenc}
\usepackage[T1]{fontenc}
\title{Scientific Reports Title to see here}

\author[1,*]{Alice Author}
\author[2]{Bob Author}
\author[1,2,+]{Christine Author}
\author[2,+]{Derek Author}
\affil[1]{Affiliation, department, city, postcode, country}
\affil[2]{Affiliation, department, city, postcode, country}

\affil[*]{corresponding.author@email.example}

\affil[+]{these authors contributed equally to this work}

%\keywords{Keyword1, Keyword2, Keyword3}

\begin{abstract}
1. Grenville Loop is younger than previously thought

2. Grenville Loop could be equivalent to the Sveconorwegian Loop

3. Large-scale differential plate motion that has been inferred from the Grenville Province can be better explained as differential uplift caused by delayed magnetic remanence acquisition. This is evident when updated closure and unblocking temperature for thermochronological and magnetic minerals, respectively, are applied to a cooling rate model of the Grenville orogen. 
\end{abstract}
\begin{document}

\flushbottom
\maketitle
% * <john.hammersley@gmail.com> 2015-02-09T12:07:31.197Z:
%
%  Click the title above to edit the author information and abstract
%
\thispagestyle{empty}

\noindent Please note: Abbreviations should be introduced at the first mention in the main text – no abbreviations lists. Suggested structure of main text (not enforced) is provided below.

\section*{Introduction}

The most rapid plate motion in Earth history was recorded on the North American plate (Laurentia) leading up to the ca. 1080-980 Ma Himalayan-scale Grenvillian orogeny (Swanson-Hysell et al., 2019). Better constraints on the paleogeography of this major continent-continent collision can thus provide critical insight into Earth's geodynamic evolution. However, determining relative plate motion between different tectonic elements of the Grenville Orogen  is complicated by uncertainty in calibrating the timing of magnetic remanance acquisition in rocks that formed within the hot, thickened crust of the orogen.
In spite of this challenge, paleomagnetic data from the lower to middle crust of the Grenville Orogen in eastern North America have been interpreted to record primary Grenvillian paleomagnetic poles (Warnock et al., 2000; other refs) and used to interpret up to 4000 km of relative plate motion between adjacent Grenvillian terranes (e.g., Halls 2015). Alternatively, it has been proposed that differences in paleomagnetic poles between Grenvillian terranes could post-date Grenvillian accretionary events and instead be the signature of differential uplift (Brett and Dunlop, 2008). 

Critical to the interpretation of paleomagnetic data across the Grenville Orogen are thermochronological data that have been used to calibrate cooling rates and the timing of magnetic remanence acquisition. Updated Pb diffusion systematics for thermochronometers have not been adequately incorporated into cooling rate models of the Grenville Orogen and its paleomagnetic-bearing rocks (Mezger references; Brown and McEnroe, 2012; other refs). In the case of rutile, previous determinations of its closure temperature were miscalibrated and resulted in estimates that were 100 to 200 \textdegree C lower than updated determinations of its closure temperature (Cherniak, 2000; Kooijman..., Smye et al., 2018). The closure temperature of thermochronometers and the unblocking temperature of magnetic carriers are both sensitive to cooling rate and other mineral characteristics, which are additional parameters that we incorporate into an updated model.

The Introduction section, of referenced text\cite{Figueredo:2009dg} expands on the background of the work (some overlap with the Abstract is acceptable). The introduction should not include subheadings.


% paleogeography reconstruction major component of reconstrucitng past earth history and inner earth evolution 



% paleomagnetism has been one major tool used for providing numerical constraints on past plate position and orientations.



% The late Mesoproterozoic was a time of large-scale tectonic activity both in the inte- rior and on the margins of Laurentia—most notably the development of the Midcontinent Rift and the Grenvillian orogeny. leading to the assmbly of Supercontinent Rodinia. 



% The Grenvillian Orogeny, one major orogenesis along Laurentia margin in the Proterozoic, led to thickened crust up to 60-70 km at its culmination and 20-30 km of erosion afterwads. Peak metamorphism reached amphibolite to granulite phase (>700 C) and result in crustal scale deformation and the compressional forces propogated more than 500 km into the Laurentia interior (Jacobsville). 





% the vast spatial extent of the orogenesis and the prolonged exhumation of deep-seated rocks yielded large uncertainties of the ages of the paleomagnetic data from the Grenville province today. The challenge is to determine the age of the acqusition of the remanent magnetization of metamorphosed rocks. Such uncertainties resulted in contrasting models on the development of the Grenville Orogeny (single-plate vs. two plate models) and long-debate about the paleogeographic reconstruction during the Neoproterozoic as the Grenville rocks are one of the few records existing during the time.



% explain metamorphism resulting in complete overprint of magnetizations (during peak metamorphism) and leaving later alteration remanence (such as heating and oxidation) at low temperature during slow exhumation (for deep-seated rocks) 

% in the existing paleomgnetic record, different In each case, one paleopole falls in group A, the
%  other in group B. The superposition of A and B components in the same rock is evidence that groups A and B do not represent scatter about some single average pole but record pole positions at two quite different times.  this is often termed the Grenville Problem. 



% Three models could explain the discordant Grenville poblem. es. One attributes the Grenville orogeny to plate convergence within the Grenville province. The other two models propose a "Grenville polar loop" either predatingor post- dating the Logan loop, reflecting drift of the entire Shield but (as yet) unrecorded from rocks outside the Grenville province or its chronological and probable lithologic equivalent in the Baltic Shield. The models stand or fall according to the dates assigned to the A and B poles. For example, the two-plate model makes sense only if the B poles marking the proposed polar juncture correspond in time to the Grenville orogeny, while the A poles are older and survived thermal overprinting during the orogeny. 



% The discovery of multivectorial magnetizations in the Haliburton intrusions (Buchan & Dunlop 1973, 1976), the Morin complex (Irving, Park & Emslie 1974) and the Whitestone anorthosite (Ueno, Irving & McNutt 1974) lent impetus to the search for a pre-orogenic magnetic record.

% McWilliams - has a failed fold test which is unequivocal evidence that the A component-carried by hematite, post-date the folding during Grenvillian Orogeny.



% about haliburton intrusion pole A and B - and related to Grenville poles that have components very simialr to A - my take is that Buchan et al is wrong - that the age of the remanence do not follow A->B, but the other way around. Becase A component is predominantly taken by hematite, not magnetit, and B component are usually magnetite, but of high unblocking temperature -> near single domain. I think what happened was that the parent rocks were completely overprinted by Grenvillian orogeny heating during 1050 Ma ish, but chemical alteration, namely oxidation, happened during slow exhumation/cooling in the deep crust susequent to the orogeny, causing oxidation of primary magnetite into ilmenite and hematite - which is often reported in the thin section eptrographic examinations - forming CRM of later-grown hematite (at temperatures below 527 C). Therefore, B is your Grenville pole and A is your later Neoproterozoic pole!



% I think C component is Taconic orogeny thermal overprint or hydrothermal activity - thin section can tell - but based on Dunlop 1985 440 Ma data from plag this is likely the case - so C component is not Grenville age at all.



% I think we should date the hematite in the A component Grenville pmag


\section*{Results}

Up to three levels of \textbf{subheading} are permitted. Subheadings should not be numbered.

\subsection*{Subsection}

Example text under a subsection. Bulleted lists may be used where appropriate, e.g.

\begin{itemize}
\item First item
\item Second item
\end{itemize}

\subsubsection*{Third-level section}
 
Topical subheadings are allowed.

\section*{Discussion}

The Discussion should be succinct and must not contain subheadings.

\section*{Methods}

Topical subheadings are allowed. Authors must ensure that their Methods section includes adequate experimental and characterization data necessary for others in the field to reproduce their work.

\bibliography{sample}

\noindent LaTeX formats citations and references automatically using the bibliography records in your .bib file, which you can edit via the project menu. Use the cite command for an inline citation, e.g.  \cite{Hao:gidmaps:2014}.

For data citations of datasets uploaded to e.g. \emph{figshare}, please use the \verb|howpublished| option in the bib entry to specify the platform and the link, as in the \verb|Hao:gidmaps:2014| example in the sample bibliography file.

\section*{Acknowledgements (not compulsory)}

Acknowledgements should be brief, and should not include thanks to anonymous referees and editors, or effusive comments. Grant or contribution numbers may be acknowledged.

\section*{Author contributions statement}

Must include all authors, identified by initials, for example:
A.A. conceived the experiment(s),  A.A. and B.A. conducted the experiment(s), C.A. and D.A. analysed the results.  All authors reviewed the manuscript. 

\section*{Additional information}

To include, in this order: \textbf{Accession codes} (where applicable); \textbf{Competing interests} (mandatory statement). 

The corresponding author is responsible for submitting a \href{http://www.nature.com/srep/policies/index.html#competing}{competing interests statement} on behalf of all authors of the paper. This statement must be included in the submitted article file.

\begin{figure}[ht]
\centering
\includegraphics[width=\linewidth]{stream}
\caption{Legend (350 words max). Example legend text.}
\label{fig:stream}
\end{figure}

\begin{table}[ht]
\centering
\begin{tabular}{|l|l|l|}
\hline
Condition & n & p \\
\hline
A & 5 & 0.1 \\
\hline
B & 10 & 0.01 \\
\hline
\end{tabular}
\caption{\label{tab:example}Legend (350 words max). Example legend text.}
\end{table}

Figures and tables can be referenced in LaTeX using the ref command, e.g. Figure \ref{fig:stream} and Table \ref{tab:example}.

\end{document}